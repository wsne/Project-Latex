\section{Applications}
Currently in our sensor network the temperature sensor mote only unicast it's results to the sink. Depending on the application we could have made several other network architectures and protocols. We'll present some small examples where we think that other protocols and/or architectures would yield some improvement of the usage.
\subsection{Data centric addressing}
 One could imagine that the sensor network should support some kind of query of temperatures. For example \emph{Where does the temperature exceed 25 degress}. If it for example was a sensor network in a office building which monitors the temperature to control the air condition system. In this case a conten-tbased addressing\cite[p.~194]{karl2007protocols} might be an advantage. Here the family of SPIN \cite[p.~335-336]{karl2007protocols} protocols could come to use. Here the temperature motes would send an advertisement to it's neighbours that the temperature average have change for example it have risen. Then the interesting motes, which controls the air condition might want to know how much it has changed. So they would send a $REQ$, and receive the data. In this case data aggregation may not be usefull, because of the network structure. The data aggregation could be usefull if for example the system should support a query of the kind what's the average temperature on this floor, or this building. Then each mote in the network hierarchy could aggregate the data by averaging every result, and thereby save some energy.
 \subsection{Geographic addressing}
This example could also apply for another addressing mote. As for example the geographic addressing \cite[p.~195]{karl2007protocols}. Here the aircondition mote mainly would be interested in the temperature in it's current room. To see if there's a big difference in maybe two opposite room ends. And then it could for example adjust the fans according to the localization of the two motes. In this example the motes aren't moving and therefore localization won't be a problem.

