\section{Patching the tinyos-core}

\subsubsection{Testing acknowledgement}
One of the bigger problems that we ran into when doing the project was: ``How
do we test ACKs?''

TinyOS detect duplicate packets and discards them.

\begin{quote}
To detect duplicate packets, a sequence number byte can be used within the
packet to verify against previously received packets. If the source address and
sequence number of a newly received packet matches that of a previously
received packet, then the newly received packet is a duplicate and may be
dumped.\cite[chapter 2.2]{PaketLink}
\end{quote}


This meant that we were not
able to test if our mote resend the packages if they did not receive an
acknowledgement. However we wanted to test rapidly sending the
same packet. 

\subsubsection{Patch TinyOS}

To make sure that we could se the packages at the basestation, but that the
temperature-mote would still retransmit the package we set the compile flag to
the preprocessor: \texttt{CFLAGS += -DCC2420\_NO\_ACKNOWLEDGEMENTS}.

Next step is to patch the following files: 

\begin{itemize}
	\item /opt/tinyos-2.x/tos/chips/cc2420/unique/DummyReceiveP.nc
	\item /opt/tinyos-2.x/tos/chips/cc2420/unique/UniqueReceive.h
	\item /opt/tinyos-2.x/tos/chips/cc2420/unique/UniqueReceive.nc
	\item /opt/tinyos-2.x/tos/chips/cc2420/unique/UniqueReceiveP.nc
\end{itemize}

Se the content of this patch on page~\siteref{src:uniqRecP} and onward. This
patch adds enables the preprocessor to take a flag more. We set this compile
flag: \texttt{CFLAGS+=-DRECEIVE\_HISTORY\_SIZE=0}

\subsection{UART / PC Communication}
Skriv om brug af printf
	
