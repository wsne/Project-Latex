\section{Extensions of Project}
Here we described some extensions of the project that we've though of. Some are implemented and others are just thoughts.
\subsection{Multiple motes}
 If there are multiple sensing motes, they send their sensing data to the sink.
\subsection{Plotting} Plot the temperature and the time at some kind of a graph. It could be the sink which handles the plotting, but maybe also the sensor could talk to the PC over the uart to be able to plot offline.
\subsection{Temperature average} The sink averages the temperature from the sensors before it's shown in the plotting or terminal.
\subsection{Temperature average sensor} The sensor motes averages $n$ temperatures before sending to the sink to improve energy efficiency of the application.
\subsection{Temperature average improved} Motes averages temperature sends number of recorded temperatures which indicates what weight the sink should use to make a correct average.
\subsection{Temperature motes} One extension could be that the temperature motes could receive some settings for example number of retries, configure the duty cycle, what version of the software and so on. But currently it doesn't receive due to energy efficiency. 
\subsection{Dynamic transmission power} One interesting strategy for retransmitting packages could for example adjusting the transmission power according to the number of retransmitments. For example increasing the transmit power in the number of retries. One problem with this extension is that we can't see how many retries there is with a current message because the tinyos is handling this for us. But another problem with this strategy is that the sink may at some point have problems with congestion. And then all the sensor would increase their transmit power and thereby increasing the noise in the network. You could also implement some congestion control to prevent this or take the congestion into consideration when increasing the transmit power.
